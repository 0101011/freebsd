\documentclass{beamer}

\usepackage[utf8]{inputenc}
\usepackage[russian]{babel}
\usepackage{tikz}
\usepackage{adjustbox}
\usepackage{url}
\usepackage{array}
\usepackage{xcolor}
\usepackage{listings}
\usepackage{verbatim}
\usepackage{ifthen}

\usetikzlibrary{positioning}
\usetikzlibrary{shapes}
\usetikzlibrary{decorations.pathmorphing}

\definecolor{yellowgreen}{HTML}{D0F000}
\definecolor{man}{HTML}{FF5E7C}
\definecolor{Agrey}{HTML}{AAAAAA}
\definecolor{Bgrey}{HTML}{BBBBBB}
\definecolor{Cgrey}{HTML}{CCCCCC}
\definecolor{Dgrey}{HTML}{DDDDDD}
\definecolor{Egrey}{HTML}{EEEEEE}

\definecolor{struct}{HTML}{999999}

% trace
\setbeamercolor{trace}{fg=black,bg=Cgrey}

% source
\setbeamercolor{source}{fg=black,bg=Cgrey}

% terminal
\setbeamercolor{terminal}{fg=white,bg=black}

% editor
\setbeamercolor{editor}{fg=black,bg=Cgrey}

% manual pages
\newcommand{\Man}[2]{
  \colorbox{white}{\color{man}#1(#2)}
}
\setbeamercolor{manref}{fg=black,bg=man}
\newcommand{\manlabel}[1]{
  \begin{beamercolorbox}[rounded=true,shadow=true,sep=0pt,colsep=0pt]{manref}
  \small{#1}
  \end{beamercolorbox}
}

% Source reference box
\setbeamercolor{srcref}{fg=black,bg=yellowgreen}
\newcommand{\srcref}[1]{
  \begin{beamercolorbox}[rounded=true,shadow=true,sep=0pt,colsep=0pt]{srcref}
  \tiny{Source code reference:}\footnotesize{ #1}
  \end{beamercolorbox}
}

% Shell command
\newcommand{\shellcmd}[1]{
  \begin{beamercolorbox}[rounded=true,shadow=true,sep=0pt,colsep=0pt]{terminal}
  \small{#1}
  \end{beamercolorbox}
}

\tikzset { struct/.style = {
		draw, thick,
                rectangle split,
		rectangle split part fill={struct!50, white!50},
} }

%
% Footnotes for source/manual references.
%
\newcommand\ManRefSw{}
\newcommand\SrcRefSw{}
\makeatletter
\newcommand*\FootReferences[2]{
	\renewcommand\ManRefSw{#1}
	\renewcommand\SrcRefSw{#2}
	\beamer@calculateheadfoot
}
\makeatother
\setbeamertemplate{footline}{%
\ifthenelse{\NOT \equal{\ManRefSw}{}} {
	\begin{beamercolorbox}
	    [wd=\paperwidth,ht=2.25ex,dp=1ex,left]{manref}%
		\hspace*{1em}Manual page(s): \ManRefSw
	\end{beamercolorbox}
  }
  % else
  {}
\ifthenelse{\NOT \equal{\SrcRefSw}{}} {
	\begin{beamercolorbox}
	    [wd=\paperwidth,ht=2.25ex,dp=1ex,left]{srcref}%
		\hspace*{1em}Source code reference: \SrcRefSw
	\end{beamercolorbox}
  }
  % else
  {}
}


\title{Input/Output system: files and VFS}

\begin{document}

\begin{frame}
\titlepage
\end{frame}


\begin{frame}
\frametitle{read(2), readv(2), write(2), writev(2)}
\begin{itemize}
\onslide <1-> {
  \item {Syscall handler fills in \emph{struct uio}.}
}
\onslide <2-> {
  \item {Generic descriptor read/write handler 
	 resolves file descriptor to \emph{struct file}.}
  \item {Method from \emph{struct file} is called: fo\_read or fo\_write.}
}
\onslide <3-> {
  \item {Vnode method passes \emph{uio} into VFS(9)}
}
\onslide <4-> {
  \item {Eventually \emph{struct bio} reaches GEOM(4) \ldots}
  \item {\ldots and a disk driver}
}
\end{itemize}
\end{frame}


\FootReferences{uiomove(9)}{sys/sys/uio.h}
\begin{frame}[fragile]
\frametitle{The user input/output structure}
\begin{verbatim}
struct uio {
  struct  iovec *uio_iov;    /* scatter/gather list */
  int     uio_iovcnt;        /* length of scatter/gather list */
  off_t   uio_offset;        /* offset in target object */
  ssize_t uio_resid;         /* remaining bytes to copy */
  enum    uio_seg uio_segflg;/* address space */
  enum    uio_rw uio_rw;     /* operation */
  struct  thread *uio_td;    /* owner */
};
\end{verbatim}
\end{frame}


\FootReferences{}{sys/kern/sys\_generic.c, sys/kern/kern\_descrip.c}
\begin{frame}
\only<1> {
  \frametitle{Resolving descriptor to \emph{struct file}}
}
\only<2> {
  \frametitle{Resolving descriptor to \emph{struct file}\\
	      \ldots and \emph{fo\_write} calling method on it}
}
\onslide<1-> {
  sys/sys\_generic.c:kern\_writev():
  \srcline {%
	fget\_write(td, fd, cap\_rights\_init(\&rights, CAP\_WRITE), \&file);
  }
}
\onslide<2-> {
  sys/sys\_generic.c:dofilewrite():
  \srcline {%
	fo\_write(fp, auio, td->td\_ucred, flags, td);
  }
}
\end{frame}


\FootReferences{}{sys/sys/file.h}
\begin{frame}[fragile]
\frametitle{File methods}
\small\begin{verbatim}
typedef int fo_rdwr_t(struct file *fp, struct uio *uio,
  struct ucred *active_cred, int flags,
  struct thread *td);
...
struct fileops {
  fo_rdwr_t       *fo_read;
  fo_rdwr_t       *fo_write;
  fo_truncate_t   *fo_truncate;
  fo_ioctl_t      *fo_ioctl;
  fo_poll_t       *fo_poll;
  fo_kqfilter_t   *fo_kqfilter;
  fo_stat_t       *fo_stat;
  fo_close_t      *fo_close;
  fo_chmod_t      *fo_chmod;
  fo_chown_t      *fo_chown;
  fo_sendfile_t   *fo_sendfile;
  fo_seek_t       *fo_seek;
};
\end{verbatim}
\end{frame}


\FootReferences{}{sys/sys/file.h}
\begin{frame}[fragile]
\frametitle{Calling file operations (methods)}
\small\begin{verbatim}
static inline int
fo_write(struct file *fp, struct uio *uio,
  struct ucred *cred, int flags, struct thread *td)
{

    return ((*fp->f_ops->fo_write)(fp, uio, cred,
      flags, td));
}
\end{verbatim}
\onslide <2-> {
\begin{itemize}
  \item{Any \emph{struct file} has a valid f\_ops pointer.}
  \item{Dead file's f\_ops point at \&badfileops.}
\end{itemize}
}
\end{frame}


\FootReferences{vnode(9)}{sys/kern/vfs\_vnops.c}
\begin{frame}[fragile]
\frametitle{a vnode(9) backed file}
\srcline{%
  fp->f\_type == DTYPE\_VNODE\\
  fp->f\_ops == \&vnops\\
  fp->f\_vnode != NULL
}
\small\begin{verbatim}
struct vnode {
  ...
  enum    vtype v_type;         /* vnode type */
  struct  vop_vector *v_op;     /* vnode operations vector */
  void    *v_data;              /* private data for fs */
  struct  mount *v_mount;       /* ptr to vfs we are in */
  int     v_usecount;           /* ref count of users */
  int     v_holdcnt;            /* prevents recycling. */
  struct bufobj   v_bufobj;     /* Buffer cache object */
  ...
}
\end{verbatim}}
\end{frame}


\FootReferences{vnode(9), VFS(9)}
	{sys/kern/vfs\_default.c, ufs/ffs/ffs\_vnops.c, ufs/ufs/ufs\_vnops.c}
\begin{frame}
\frametitle{VFS(9) - object oriented approach to vnodes}
\only <1> {
  VOP\_READ() resolves to the first vop, that supports read.
}
\only <2> {
  VFS(9) example: ffs/ufs vops.
}
\begin{figure}
\small\begin{tikzpicture}
\only <1> {
  \node [name=vop1, struct, rectangle split parts=6] {
	\textbf{struct vop\_vector}
	\nodepart{two}struct vop\_vector *vop\_default 
	\nodepart{three}vop\_read\_t vop\_read
	\nodepart{four}vop\_write\_t vop\_write
	\nodepart{five}\ldots
	\nodepart{six}\textit{up to 70 ops}
  };
  \node [name=vop2, struct, right=of vop1.north east, anchor=north west,
	 rectangle split parts=4] {
	\textbf{struct vop\_vector}
	\nodepart{two}struct vop\_vector *vop\_default 
	\nodepart{three}vop\_strategy\_t vop\_strategy
	\nodepart{four}\ldots
  };
  \node [name=vop3, struct, below=of vop2, rectangle split parts=1] {
	\textbf{struct vop\_vector}
  };
  \draw [pointer] (vop1.two east) to [out=0, in=180] (vop2.one west);
  \draw [pointer] (vop2.two east) -- +(5mm, 0) |- (vop3);
}
\only <2> {
  \node [name=vop1, struct, rectangle split parts=5] {
	\textbf{ffs\_vnodeops}
	\nodepart{two}vop\_default 
	\nodepart{three}vop\_read = ffs\_read
	\nodepart{four}vop\_write = ffs\_write
	\nodepart{five}\ldots
  };
  \node [name=vop2, struct, right=of vop1.north east, anchor=north west,
	 rectangle split parts=4] {
	\textbf{ufs\_vnodeops}
	\nodepart{two}vop\_default 
	\nodepart{three}vop\_strategy = ufs\_strategy
	\nodepart{four}\ldots
  };
  \node [name=vop3, struct, below=of vop2, rectangle split parts=1] {
	\textbf{default\_vnodeops}
  };
  \draw [pointer] (vop1.two east) to [out=0, in=180] (vop2.one west);
  \draw [pointer] (vop2.two east) -- +(5mm, 0) |- (vop3);
}
\end{tikzpicture}
\end{figure}
\end{frame}

\end{document}
