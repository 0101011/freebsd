\documentclass{beamer}

\usepackage[utf8]{inputenc}
\usepackage[russian]{babel}
\usepackage{tikz}
\usepackage{url}
\usepackage{xcolor}
\usepackage{listings}
\usepackage{verbatim}
\usepackage{ifthen}

\usetikzlibrary{positioning}
\usetikzlibrary{shapes}
\usetikzlibrary{calc}

\definecolor{yellowgreen}{HTML}{D0F000}
\definecolor{man}{HTML}{FF5E7C}
\definecolor{Agrey}{HTML}{AAAAAA}
\definecolor{Bgrey}{HTML}{BBBBBB}
\definecolor{Cgrey}{HTML}{CCCCCC}
\definecolor{Dgrey}{HTML}{DDDDDD}
\definecolor{Egrey}{HTML}{EEEEEE}

\definecolor{struct}{HTML}{999999}

% trace
\setbeamercolor{trace}{fg=black,bg=Cgrey}

% source
\setbeamercolor{source}{fg=black,bg=Cgrey}

% terminal
\setbeamercolor{terminal}{fg=white,bg=black}

% editor
\setbeamercolor{editor}{fg=black,bg=Cgrey}

% manual pages
\newcommand{\Man}[2]{
  \colorbox{white}{\color{man}#1(#2)}
}
\setbeamercolor{manref}{fg=black,bg=man}
\newcommand{\manlabel}[1]{
  \begin{beamercolorbox}[rounded=true,shadow=true,sep=0pt,colsep=0pt]{manref}
  \small{#1}
  \end{beamercolorbox}
}

% Source reference box
\setbeamercolor{srcref}{fg=black,bg=yellowgreen}
\newcommand{\srcref}[1]{
  \begin{beamercolorbox}[rounded=true,shadow=true,sep=0pt,colsep=0pt]{srcref}
  \tiny{Source code reference:}\footnotesize{ #1}
  \end{beamercolorbox}
}

% Shell command
\newcommand{\shellcmd}[1]{
  \begin{beamercolorbox}[rounded=true,shadow=true,sep=0pt,colsep=0pt]{terminal}
  \small{#1}
  \end{beamercolorbox}
}

\tikzset { struct/.style = {
		draw, thick,
                rectangle split,
		rectangle split part fill={struct!50, white!50},
} }

%
% Footnotes for source/manual references.
%
\newcommand\ManRefSw{}
\newcommand\SrcRefSw{}
\makeatletter
\newcommand*\FootReferences[2]{
	\renewcommand\ManRefSw{#1}
	\renewcommand\SrcRefSw{#2}
	\beamer@calculateheadfoot
}
\makeatother
\setbeamertemplate{footline}{%
\ifthenelse{\NOT \equal{\ManRefSw}{}} {
	\begin{beamercolorbox}
	    [wd=\paperwidth,ht=2.25ex,dp=1ex,left]{manref}%
		\hspace*{1em}Manual page(s): \ManRefSw
	\end{beamercolorbox}
  }
  % else
  {}
\ifthenelse{\NOT \equal{\SrcRefSw}{}} {
	\begin{beamercolorbox}
	    [wd=\paperwidth,ht=2.25ex,dp=1ex,left]{srcref}%
		\hspace*{1em}Source code reference: \SrcRefSw
	\end{beamercolorbox}
  }
  % else
  {}
}


\title{Processes and threads}

\begin{document}

\begin{frame}
\titlepage
\end{frame}

\FootReferences{}{sys/sys/proc.h}
\begin{frame}
\frametitle{Process structure}
\note {
	- A process is a virtual machine that kernel creates for
	  an application. It provides VM and CPU resources.\\
	- The slide is a minimal layout of a UNIX process. Actual
	  size of struct proc is > 1Kb.
}
\begin{tikzpicture}[every node/.style={draw, node distance=10mm}]
  \node [name=proc, struct, inner sep=2mm, rectangle split parts = 8] {
    \textbf{struct proc}
    \nodepart{two} struct vmspace *p\_vmspace
    \nodepart{three} struct sysentvec *p\_sysent
    \nodepart{four} struct filedesc *p\_fd
    \nodepart{five} struct ucred *p\_ucred
    \nodepart{six} struct plimit *p\_limit
    \nodepart{seven} \ldots
    \nodepart{eight} TAILQ\_HEAD() p\_threads
  };

  \node[name=vm, right=of proc.two east, rounded corners]
     { VM space };
  \draw [->] (proc.two east) -- (vm);
  \node[name=sysent, right=of proc.three east, rounded corners]
     { syscall vector };
  \draw [->] (proc.three east) -- (sysent);
  \node[name=fdesc, right=of proc.four east, rounded corners]
     { file descriptor table };
  \draw [->] (proc.four east) -- (fdesc);
  \node[name=rlim, right=of proc.five east, rounded corners]
     { resource limits };
  \draw [->] (proc.five east) -- (rlim);
  \node[name=cred, right=of proc.six east, rounded corners]
     { credentials };
  \draw [->] (proc.six east) -- (cred);

  % threads
  \node [name=thread1, struct, rectangle split parts = 1,
         node distance=5mm, below right=of proc]
    { \textbf{struct thread} };
  \node [name=thread2, struct, rectangle split parts = 1,
         node distance=5mm, right=of thread1]
    { \textbf{struct thread} };
  \draw [->] (proc.eight east) to [out=0, in=180] (thread1.west);
  \draw [->] (thread1) -- (thread2);
\end{tikzpicture}
\end{frame}


\begin{frame}
\frametitle{p\_state - the process state}
\begin{itemize}
  \item{PRS\_NEW}
  \item{PRS\_NORMAL}
  \item{PRS\_ZOMBIE}
\end{itemize}
\end{frame}


\begin{frame}
\frametitle{Thread structure}
\note {
	- Thread is a executing entity within a process. It includes all
	  the data that can't be shared.\\
	- A minimal struct thread layout. Actual size > 1 Kb.
}
\begin{tikzpicture}[every node/.style={draw, node distance=10mm}]
  \node [name=thread, struct, inner sep=2mm, rectangle split parts = 6] {
    \textbf{struct thread}
    \nodepart{two} struct proc *td\_proc
    \nodepart{three} struct proc *td\_sched
    \nodepart{four} struct pcb *td\_pcb
    \nodepart{five} struct vm\_object *td\_kstack\_obj
    \nodepart{six} \ldots
  };

  \node [name=proc, right=of thread.two east, struct, rectangle split parts = 1]
    { \textbf{struct proc} };
  \draw [->] (thread.two east) -- (proc);
  \node [name=sched, right=of thread.three east, rounded corners]
    { scheduling info };
  \draw [->] (thread.three east) -- (sched);
  \node [name=pcb, right=of thread.four east, rounded corners]
    { thread control block (MD) };
  \draw [->] (thread.four east) -- (pcb);
  \node [name=kstack, right=of thread.five east, rounded corners]
    { kernel stack };
  \draw [->] (thread.five east) -- (kstack);
\end{tikzpicture}
\end{frame}


\FootReferences{}{sys/sys/proc.h, sys/kern/kern\_thread.c,
		  sys/kern/sched\_ule.c}
\begin{frame}
\frametitle{td\_state - the thread state}
\begin{itemize}
  \item{TDS\_INACTIVE}
  \item{TDS\_CAN\_RUN}
  \item{TDS\_RUNQ}
  \item{TDS\_RUNNING}
  \item{
    TDS\_INHIBITED\\
    \onslide<2> {
	Combination of inhibitors in td\_inhibitors:
	\begin{itemize}
		\item{TDI\_SUSPENDED}
		\item{TDI\_SLEEPING}
		\item{TDI\_SWAPPED}
		\item{TDI\_LOCK}
		\item{TDI\_IWAIT}
	\end{itemize}
    }
  }
\end{itemize}
\end{frame}


\FootReferences{}{}
\begin{frame}
\frametitle{Historic note: M:N vs 1:1}
\begin{columns}[c]
  \begin{column}{.46\paperwidth}
    \begin{figure}\begin{tikzpicture}
	\node [name=thread, struct, rectangle split parts = 1]
		{ \textbf{struct thread} };
	\draw ($(thread) + (-.33\paperwidth, 2cm)$) --
	    node[very near end, above] { process }
	    node[very near end, below] { kernel } 
	    ($(thread) + (.13\paperwidth, 2cm)$);
	\node [name=uthr1, draw, ellipse, fill=white,
	    node distance=0, anchor=center,
	    above left=.3\paperheight and -.08\paperwidth of thread.center,
	    text width=2cm, align=center]
		{ user space scheduling entity };
	\draw [->] (uthr1) -- (thread);
	\node [name=uthr2, draw, ellipse, fill=white,
	    node distance=0, anchor=center,
	    above left=.25\paperheight and .1\paperwidth of thread.center,
	    text width=2cm, align=center]
		{ user space scheduling entity };
	\node [name=thread2, struct, rectangle split parts = 1,
	       node distance=0, above left=3mm and -10mm of thread]
		{ \textbf{struct thread} };
	\draw [->] (uthr2) -- (thread2);

	\node [name=pthread1, node distance=.6\paperheight, above=of thread]
		{ pthread\_t };
	\node [name=pthread2, node distance=0, left=.1\paperwidth of pthread1]
		{ pthread\_t };
	\node [name=pthread21, node distance=0, below right=of pthread2]
		{ pthread\_t };
	\node [name=pthread22, node distance=0, below right=of pthread21]
		{ pthread\_t };
	\node [name=pthread23, node distance=0, below left=of pthread21]
		{ pthread\_t };

	\draw [->] (pthread23) -- (uthr2);
	\draw [->] (pthread22) -- (uthr1);
	\draw [->] (pthread21) -- (uthr1);
	\draw [->] (pthread1) to [out=270,in=100]
		   (node cs:name=uthr1,angle=100);
	\draw [->] (pthread2.south) to [out=350,in=80]
		   (node cs:name=uthr2,angle=80);
    \end{tikzpicture}\end{figure}
  \end{column}
  \begin{column}{.08\paperwidth}
  \LARGE{VS}
  \end{column}
  \begin{column}{.46\paperwidth}
    \begin{figure}\begin{tikzpicture}
	\node [name=thread, struct, rectangle split parts = 1]
		{ \textbf{struct thread} };
	\draw ($(thread) + (-.23\paperwidth, 2cm)$) --
	    node[very near end, above] { process }
	    node[very near end, below] { kernel } 
	    ($(thread) + (.23\paperwidth, 2cm)$);
	\node [name=pthread, node distance=.6\paperheight, above=of thread]
		{ pthread\_t };
	\draw [->] (pthread) -- (thread);
    \end{tikzpicture}\end{figure}
  \end{column}
\end{columns}
\end{frame}


\begin{frame}
\frametitle{Historic note: M:N vs 1:1}
\only<1> {
  \begin{itemize}
    \item{1993 Solaris 2.2 introduces multithreading via non-standard API}
    \item{1995 Solaris 2.5 supports POSIX threads API, \textbf{M:N model}}
    \item{2000 FreeBSD starts SMPng project targeted at \textbf{M:N}}
    \item{2000 Solaris 8 provides \textbf{1:1} threading as option}
    \item{2001 Linux starts multithreading project targeted at \textbf{1:1}
  	with \textbf{M:N} as experimental option}
    \item{2001 Mac OS X released with \textbf{1:1} threading}
    \item{2002 Solaris 9 uses \textbf{1:1} threading as default}
    \item{2003 Linux abandons experimental \textbf{M:N} project}
    \item{2004 FreeBSD 5.3 released with Kernel Scheduling Entities
  	(\textbf{M:N}) threading as default, \textbf{1:1} optional}
    \item{2004 NetBSD 2.0 released with Scheduler Activations (\textbf{M:N})}
    \item{2004 Linux 2.6 released with \textbf{1:1} threading}
  \end{itemize}
}
\only<2-> {
  \begin{itemize}
    \item{2005 FreeBSD moves focus to \textbf{1:1}}
    \item{2007 NetBSD abandons SA (\textbf{M:N})}
    \item{2008 FreeBSD abandons KSE (\textbf{M:N})}
    \item{2008 FreeBSD 7.0 released with \textbf(1:1) as default,
  	\textbf{M:N} optional}
    \item{2009 NetBSD 5.0 released with \textbf{1:1} as only option}
    \item{2009 FreeBSD 8.0 released with \textbf{1:1} as only option}
    \onslide<3>\item{2009 Windows 7 introduces User-Mode Scheduling}
  \end{itemize}
}
\end{frame}

\end{document}
