\documentclass{beamer}

\usepackage[utf8]{inputenc}
\usepackage[russian]{babel}
\usepackage{tikz}
\usepackage{url}
\usepackage{xcolor}
\usepackage{listings}
\usepackage{verbatim}
\usepackage{ifthen}
\usepackage{bytefield}
\usepackage{wasysym}

\usetikzlibrary{positioning}
\usetikzlibrary{shapes}
\usetikzlibrary{arrows}
\usetikzlibrary{decorations.text}
\usetikzlibrary{patterns}
\usetikzlibrary{chains}
\usetikzlibrary{scopes}
\usetikzlibrary{calc}

\definecolor{yellowgreen}{HTML}{D0F000}
\definecolor{man}{HTML}{FF5E7C}
\definecolor{Agrey}{HTML}{AAAAAA}
\definecolor{Bgrey}{HTML}{BBBBBB}
\definecolor{Cgrey}{HTML}{CCCCCC}
\definecolor{Dgrey}{HTML}{DDDDDD}
\definecolor{Egrey}{HTML}{EEEEEE}

\definecolor{struct}{HTML}{999999}

% trace
\setbeamercolor{trace}{fg=black,bg=Cgrey}

% source
\setbeamercolor{source}{fg=black,bg=Cgrey}

% terminal
\setbeamercolor{terminal}{fg=white,bg=black}

% editor
\setbeamercolor{editor}{fg=black,bg=Cgrey}

% manual pages
\newcommand{\Man}[2]{
  \colorbox{white}{\color{man}#1(#2)}
}
\setbeamercolor{manref}{fg=black,bg=man}
\newcommand{\manlabel}[1]{
  \begin{beamercolorbox}[rounded=true,shadow=true,sep=0pt,colsep=0pt]{manref}
  \small{#1}
  \end{beamercolorbox}
}

% Source reference box
\setbeamercolor{srcref}{fg=black,bg=yellowgreen}
\newcommand{\srcref}[1]{
  \begin{beamercolorbox}[rounded=true,shadow=true,sep=0pt,colsep=0pt]{srcref}
  \tiny{Source code reference:}\footnotesize{ #1}
  \end{beamercolorbox}
}

% Shell command
\newcommand{\shellcmd}[1]{
  \begin{beamercolorbox}[rounded=true,shadow=true,sep=0pt,colsep=0pt]{terminal}
  \small{#1}
  \end{beamercolorbox}
}

\tikzset { struct/.style = {
		draw, thick,
                rectangle split,
		rectangle split part fill={struct!50, white!50},
} }

%
% Footnotes for source/manual references.
%
\newcommand\ManRefSw{}
\newcommand\SrcRefSw{}
\makeatletter
\newcommand*\FootReferences[2]{
	\renewcommand\ManRefSw{#1}
	\renewcommand\SrcRefSw{#2}
	\beamer@calculateheadfoot
}
\makeatother
\setbeamertemplate{footline}{%
\ifthenelse{\NOT \equal{\ManRefSw}{}} {
	\begin{beamercolorbox}
	    [wd=\paperwidth,ht=2.25ex,dp=1ex,left]{manref}%
		\hspace*{1em}Manual page(s): \ManRefSw
	\end{beamercolorbox}
  }
  % else
  {}
\ifthenelse{\NOT \equal{\SrcRefSw}{}} {
	\begin{beamercolorbox}
	    [wd=\paperwidth,ht=2.25ex,dp=1ex,left]{srcref}%
		\hspace*{1em}Source code reference: \SrcRefSw
	\end{beamercolorbox}
  }
  % else
  {}
}


\title{Memory management in FreeBSD}

\tikzset {
	entry/.style={draw, thick, on chain, text width=.25\paperwidth,
		      align=center},
	large/.style={entry, minimum height=.15\paperheight},
	noentry/.style={entry, fill=gray, shading=axis, shading angle=45},
	nolarge/.style={large, fill=gray, shading=axis, shading angle=45},
	page/.style={draw, thick, node distance=3mm},
	vmmap/.style={draw, thick, rounded corners},
	vmentry/.style={draw, thick, rounded corners},
	vmobject/.style={draw, thick, rounded corners, text width=7ex },
	pointer/.style={->, thick, rounded corners},
}

\begin{document}

\begin{frame}
\titlepage
\end{frame}


\begin{frame}
\frametitle{Process (static) address space}
\begin{figure}
\begin{tikzpicture}[start chain=going below, node distance=0mm]
    \node [name=kernel, nolarge] { kernel };
    \node [name=argv, entry] { argv, envp };
    \node [name=stack, large] { stack };
    \node [name=unmap, nolarge] { not mapped memory };
    \node [name=heap, large] { heap };
    \node [name=data, entry] { initialized data };
    \node [name=text, entry] { program text };
    \node [name=null, noentry] { };

    \draw [<-] (kernel.north west) --
	node [below] {\scriptsize{0xff \ldots ff}} +(-1.5cm, 0);
    \draw [<-] (null.south west) --
	node [above] {\scriptsize{0x00}} +(-1.5cm, 0);
\end{tikzpicture}
\end{figure}
\end{frame}

\FootReferences{procstat(1)}{}
\begin{frame}
\frametitle{Process (static) address space}
Try this out:
\shellcmd{%
\# /rescue/cat \&\\
\lbrack1\rbrack 48141\\
\# procstat -v \$!\\
\begin{tabular}{rrrrrrrrrrr}
  PID &            START &              END & PRT  & RES & PRES & REF & SHD &   FL & TP & PATH \\
48141 &         0x400000 &         0x96d000 & r-x  & 677 &    0 &   4 &   2 & CN-- & vn & /rescue/chio \\
48141 &         0xb6d000 &         0xb94000 & rw-  &  38 &    0 &   1 &   0 & C--- & vn & /rescue/chio \\
48141 &         0xb94000 &         0xdc8000 & rw-  &  21 &    0 &   1 &   0 & ---- & df & \\
48141 &      0x800c00000 &      0x801400000 & rw-  &  81 &    0 &   1 &   0 & ---- & df & \\
48141 &   0x7fffffbfe000 &   0x7fffffbff000 & ---  &   0 &    0 &   0 &   0 & ---- & -- & \\
48141 &   0x7ffffffdf000 &   0x7ffffffff000 & rw-  &   4 &    0 &   1 &   0 & ---D & df & \\
48141 &   0x7ffffffff000 &   0x800000000000 & r-x  &   0 &    0 &  67 &   0 & ---- & ph & \\
\end{tabular}
}
\end{frame}


\FootReferences{}{}
\begin{frame}
\frametitle{Stack grows implicitly}
\begin{columns}
\begin{column}{.4\paperwidth}
  \begin{figure}
  \begin{tikzpicture}[start chain=going below, node distance=0mm]
    \node [name=argv, entry] { argv, envp };
    \node [name=stack, large] { stack };
    \node [name=unmap, nolarge] { not mapped memory };
    \node [name=heap, large] { heap };
    \node [name=data, entry] { initialized data };
    \node [name=text, entry] { program text };
    \node [name=null, noentry] { };

    \draw (stack.south west) ++(0cm, -2mm) [<-] --
	node [above] {\scriptsize{deref!}} +(-1.5cm, 0);
  \end{tikzpicture}
  \end{figure}
\end{column}
\begin{column}{.2\paperwidth}
  \begin{tikzpicture}
    \draw [->, thick] (0,0) -- node [above] {\small{page fault}}
	+(.2\paperwidth,0);
  \end{tikzpicture}
\end{column}
\begin{column}{.4\paperwidth}
  \begin{figure}
  \begin{tikzpicture}[start chain=going below, node distance=0mm]
    \node [name=argv, entry] { argv, envp };
    \node [name=stack, large, minimum height=.20\paperheight] { stack };
    \node [name=unmap, nolarge, minimum height=.10\paperheight]
	{ not mapped memory };
    \node [name=heap, large] { heap };
    \node [name=data, entry] { initialized data };
    \node [name=text, entry] { program text };
    \node [name=null, noentry] { };

    \draw [thick] (stack.center) ++(-1cm,0) node (mark1) {}
	[->] -- (mark1 |- stack.south);
    \draw [thick] (stack.center) ++(1cm,0) node (mark2) {}
	[->] -- (mark2 |- stack.south);
  \end{tikzpicture}
  \end{figure}
\end{column}
\end{columns}
\end{frame}


\FootReferences{sbrk(2)}{}
\begin{frame}
\frametitle{Heap grows explicitly (used to grow \smiley )}
\begin{columns}
\begin{column}{.4\paperwidth}
  \begin{figure}
  \begin{tikzpicture}[start chain=going below, node distance=0mm]
    \node [name=argv, entry] { argv, envp };
    \node [name=stack, large] { stack };
    \node [name=unmap, nolarge] { not mapped memory };
    \node [name=heap, large] { heap };
    \node [name=data, entry] { initialized data };
    \node [name=text, entry] { program text };
    \node [name=null, noentry] { };
  \end{tikzpicture}
  \end{figure}
\end{column}
\begin{column}{.2\paperwidth}
  \begin{tikzpicture}
    \draw [->, thick] (0,0) -- node [above] {\small{sbrk(2)}}
	+(.2\paperwidth,0);
  \end{tikzpicture}
\end{column}
\begin{column}{.4\paperwidth}
  \begin{figure}
  \begin{tikzpicture}[start chain=going below, node distance=0mm]
    \node [name=argv, entry] { argv, envp };
    \node [name=stack, large] { stack };
    \node [name=unmap, nolarge, minimum height=.10\paperheight]
	{ not mapped memory };
    \node [name=heap, large, minimum height=.20\paperheight] { heap };
    \node [name=data, entry] { initialized data };
    \node [name=text, entry] { program text };
    \node [name=null, noentry] { };

    \draw [thick] (heap.center) ++(-1cm,0) node (mark1) {}
	[->] -- (mark1 |- heap.north);
    \draw [thick] (heap.center) ++(1cm,0) node (mark2) {}
	[->] -- (mark2 |- heap.north);
  \end{tikzpicture}
  \end{figure}
\end{column}
\end{columns}
\end{frame}


\FootReferences{mmap(2)}{}
\begin{frame}
\frametitle{Mapping more on heap}
\begin{columns}
\begin{column}{.4\paperwidth}
  \begin{figure}
  \begin{tikzpicture}[start chain=going below, node distance=0mm]
    \node [name=argv, entry] { argv, envp };
    \node [name=stack, large] { stack };
    \node [name=unmap, nolarge] { not mapped memory };
    \node [name=heap, large] { heap };
    \node [name=data, entry] { initialized data };
    \node [name=text, entry] { program text };
    \node [name=null, noentry] { };
  \end{tikzpicture}
  \end{figure}
\end{column}
\begin{column}{.2\paperwidth}
  \begin{tikzpicture}
    \draw [->, thick] (0,0) -- node [above] {\small{mmap(2)}}
	+(.2\paperwidth,0);
  \end{tikzpicture}
\end{column}
\begin{column}{.4\paperwidth}
  \begin{figure}
  \begin{tikzpicture}[start chain=going below, node distance=0mm]
    \node [name=argv, entry] { argv, envp };
    \node [name=stack, large] { stack };
    \node [name=unmap1, nolarge, minimum height=.05\paperheight]
	{ not mapped };
    \node [name=mmap, large, minimum height=.05\paperheight]
	{ more heap };
    \node [name=unmap2, nolarge, minimum height=.05\paperheight]
	{ not mapped };
    \node [name=heap, large] { heap };
    \node [name=data, entry] { initialized data };
    \node [name=text, entry] { program text };
    \node [name=null, noentry] { };
  \end{tikzpicture}
  \end{figure}
\end{column}
\end{columns}
\end{frame}


\FootReferences{mmap(2), malloc(3), rtld(1)}{}
\begin{frame}
\frametitle{Modern process memory map}
\begin{columns}
\begin{column}{.5\paperwidth}
  \begin{figure}
  \begin{tikzpicture}[start chain=going below, node distance=0mm]
    \tikzset {
	large/.style={entry, minimum height=.1\paperheight},
    }
    \node [name=argv, entry] { argv, envp };
    \node [name=stack, entry] { stack };
    \node [name=unmap1, nolarge] { not mapped };
    \node [name=mmap, entry] { malloc arena };
    \node [name=unmap2, nolarge] { not mapped };
    \node [name=lib, entry] { library };
    \node [name=unmap3, nolarge] { not mapped };
    \node [name=heap, entry] { ld-elf };
    \node [name=data, entry] { initialized data };
    \node [name=text, entry] { program text };
    \node [name=null, noentry] { };
  \end{tikzpicture}
  \end{figure}
\end{column}
\begin{column}{.5\paperwidth}
\begin{itemize}
  \item{program itself via mmap(2)}
  \item{dynamic libraries via mmap(2)}
  \item{malloc(3) arenas via mmap(2)}
\end{itemize}
\onslide<2-> {
    Try this out:
    \shellcmd{%
    \# /bin/cat \&\\
    \# procstat -v \$!\\
    }
    Or this:
    \shellcmd{%
    \# procstat -v \$\$\\
    }
}
\end{column}
\end{columns}
\end{frame}


\FootReferences{}{}
\begin{frame}
\frametitle{VM space}
\begin{figure}
\begin{tikzpicture}[start chain=going below, node distance=0mm]
    \tikzset {
	large/.style={entry, minimum height=.1\paperheight},
    }
    \node [name=argv, entry] { argv, envp };
    \node [name=stack, entry] { stack };
    \node [name=unmap1, nolarge] { not mapped };
    \node [name=mmap, entry] { malloc arena };
    \node [name=unmap2, nolarge] { not mapped };
    \node [name=lib, entry] { library };
    \node [name=unmap3, nolarge] { not mapped };
    \node [name=heap, entry] { ld-elf };
    \node [name=data, entry] { initialized data };
    \node [name=text, entry] { program text };
    \node [name=null, noentry] { };

    \node [name=vmspace, draw, thick, color=red, ellipse,
	  minimum width=.35\paperwidth,
	  minimum height=.8\paperheight] at (lib.north) {};
    \node [ellipse callout, draw, color=red,
	   callout absolute pointer={(node cs:name=vmspace, angle=40)},
	   node distance=5mm, right=of stack] { vm\_map };

    \node [name=vmentry, draw, thick, color=red, ellipse,
	  minimum width=.30\paperwidth,
	  minimum height=2em] at (lib.center) {};
    \node [ellipse callout, draw, color=red,
	   callout absolute pointer={(node cs:name=vmentry, angle=2)},
	   node distance=7mm, right=of unmap2] { vm\_map\_entry };
\end{tikzpicture}
\end{figure}
\end{frame}


\FootReferences{}{sys/vm/vm\_map.h}
\begin{frame}
\frametitle{Kernel representation of VM space}
\begin{figure}
\small\begin{tikzpicture}
  \node [name=vmspace, struct, rectangle split parts=3] {
	\textbf{struct vmspace}
	\nodepart{two} struct vm\_map
	\nodepart{three} struct vm\_pmap
  };
  \node [name=vmmap, struct, right=of vmspace, rectangle split parts=4] {
	\textbf{struct vm\_map}
	\nodepart{two} struct vm\_map\_entry header
	\nodepart{three} struct vm\_map\_entry \*tree
	\nodepart{four} \ldots
  };
  \node [name=entry1, struct, below=of vmspace, rectangle split parts=5] {
	\textbf{struct vm\_map\_entry}
	\nodepart{two} struct vm\_map\_entry *next
	\nodepart{three} vm\_offset\_t start
	\nodepart{four} vm\_offset\_t end
	\nodepart{five} \ldots
  };
  \node [name=entry2, struct, right=of entry1, rectangle split parts=5] {
	\textbf{struct vm\_map\_entry}
	\nodepart{two} struct vm\_map\_entry *next
	\nodepart{three} vm\_offset\_t start
	\nodepart{four} vm\_offset\_t end
	\nodepart{five} \ldots
  };

  \node [name=mark, node distance=5mm, above left=of entry1] {};
  \draw [pointer] (vmmap.two east) -- ++(5mm,0)
	|- (mark.center) |- (entry1.one west);

  \draw [->, thick] (entry1.two east) to [out=0, in=180] (entry2.one west);
  \draw [pointer] (entry2.two east) --
	++(5mm,0) -- ++(0,-1cm);
\end{tikzpicture}
\end{figure}
\end{frame}


\FootReferences{}{sys/vm/vm\_map.h, sys/vm/vm\_object.h}
\begin{frame}
\frametitle{A VM map entry is backed by an object}
\begin{figure}
\begin{tikzpicture}
  \node [name=entry, struct, rectangle split parts=4] {
	\textbf{struct vm\_map\_entry}
	\nodepart{two} \ldots
	\nodepart{three} struct vm\_object *object
	\nodepart{four} \ldots
  };
  \node [name=object, struct, right=of entry, rectangle split parts=5] {
	\textbf{struct vm\_object}
	\nodepart{two} \ldots
	\nodepart{four} union *pager
	\nodepart{three} struct vm\_radix rtree
	\nodepart{five} \ldots
  };

  \draw [->, thick] (entry.three east) to [out=0, in=180] (object.one west);

  \node [name=page1, page, below=of object] { vm\_page };
  \draw [->, thick, rounded corners] (object.three east) -- ++(1cm,0)
	|- (page1.east);

  \node [name=page2, page, below right=of page1] { vm\_page };
  \node [name=page3, page, below left=of page1] { vm\_page };
  \draw [->,thick] (page1.south) to [out=290, in=180] (page2.west);
  \draw [->,thick] (page1.south) to [out=250, in=0] (page3.east);

  \node [name=page4, page, below right=of page3] { vm\_page };
  \node [name=page5, page, below left=of page3] { vm\_page };
  \draw [->,thick] (page3.south) to [out=290, in=180] (page4.west);
  \draw [->,thick] (page3.south) to [out=250, in=0] (page5.east);

  \node [name=page6, page, below right=of page5] { vm\_page };
  \node [name=page7, page, below left=of page5] { vm\_page };
  \draw [->,thick] (page5.south) to [out=290, in=180] (page6.west);
  \draw [->,thick] (page5.south) to [out=250, in=0] (page7.east);

  \node [name=page8, page, below right=of page4] { vm\_page };
  \draw [->,thick] (page4.south) to [out=290, in=180] (page8.west);

\end{tikzpicture}
\end{figure}
\end{frame}


\FootReferences{}{sys/vm/vm\_page.h, sys/vm/vm\_page.c}
\begin{frame}
\frametitle{VM page}
  \begin{itemize}
    \item{Small structure describing one resident physical page}
    \item{All vm\_page structs are allocated at boot time}
    \item{A page belongs to one object}
  \end{itemize}
\end{frame}


\FootReferences{}{sys/vm/vm\_pager.h, sys/vm/vm\_pager.c}
\begin{frame}
\frametitle{VM objects' pager}
\begin{columns}
\begin{column}{.5\paperwidth}
  \begin{itemize}
    \item{Object serves as cache of memory page}
    \item{Pager implements particular cache in/cache out semantics}
  \end{itemize}
\end{column}
\begin{column}{.5\paperwidth}
  \begin{figure}
  \begin{tikzpicture}
     \node [struct, rectangle split parts=7] {
	\textbf{struct pagerops}
	\nodepart{two} pgo\_init\_t      *pgo\_init
	\nodepart{three} pgo\_alloc\_t     *pgo\_alloc
	\nodepart{four} pgo\_dealloc\_t   *pgo\_dealloc
	\nodepart{five} pgo\_getpages\_t  *pgo\_getpages
	\nodepart{six} pgo\_putpages\_t  *pgo\_putpages
	\nodepart{seven} pgo\_haspage\_t   *pgo\_haspage
     };
  \end{tikzpicture}
  \end{figure}
\end{column}
\end{columns}
\end{frame}


\FootReferences{}{sys/vm/vm\_object.h, sys/vm/vm\_object.c,
		  sys/vm/default\_pager.c, sys/vm/swap\_pager.c}
\begin{frame}
\frametitle{Anonymous object}
\begin{columns}
\begin{column}{.5\paperwidth}
  \begin{figure}
  \begin{tikzpicture}
      \node [name=object, struct, rectangle split parts=5] {
	\textbf{struct vm\_object}
	\only<1> { \nodepart{two} \textit{uses default pager} }
	\only<2,3> { \nodepart{two} \textit{{\color{red}uses swap pager}}}
	\nodepart{three} \ldots
	\nodepart{four} struct vm\_radix rtree
	\nodepart{five} \ldots
      };
      \node [name=page1, page, below right=of object] {vm\_page};
      \draw [pointer] (object.five east) -| (page1.north);
      \only<1,2> {
        \node [name=page2, page, node distance=0, below=of page1] {vm\_page};
        \node [name=page3, page, node distance=0, below=of page2] {vm\_page};
      }
      \only<2> {
	\node [name=swap, draw, rounded corners, fill=gray, shading=axis,
	  left=of page3, xshift=-2em]
	  {swap partition};
	\node [name=ell, draw, thick, color=red, ellipse,
	  minimum width=12ex, minimum height=4em] at (page2.south) {};
	\draw [->, thick, color=red]
	  (node cs:name=ell, angle=235) to [out=235, in=315]
	  node [above] { swap out } (swap.south);
      }
  \end{tikzpicture}
  \end{figure}
\end{column}
\begin{column}{.4\paperwidth}
  \begin{itemize}
    \item{Anonymous object usually backs vm\_map\_entry made by sbrk(),
	  or mmap(MAP\_ANON), or stack}
    \item{Default pager is a nop}
    \onslide<2-> {
      \item{In case of memory pressure default pages is changed to swap pager}
    }
    \onslide<3-> {
      \item{Object remains with swap pager forever}
    }
  \end{itemize}
\end{column}
\end{columns}
\end{frame}


\FootReferences{}{sys/vm/vm\_object.h, sys/vm/vm\_object.c,
		  sys/vm/vnode\_pager.c}
\begin{frame}
\frametitle{Vnode backed object}
\begin{columns}
\begin{column}{.5\paperwidth}
  \begin{figure}
  \begin{tikzpicture}
      \node [name=object, struct, rectangle split parts=6] {
	\textbf{struct vm\_object}
	\nodepart{two} uses vnode pager
	\nodepart{three} \ldots
	\nodepart{four} void *handle
	\nodepart{five} struct vm\_radix rtree
	\nodepart{six} \ldots
      };
      \node [name=vnode, struct, below=of object.south west, anchor=west,
	     rectangle split parts=1] {
	\textbf{struct vnode}
      };
      \draw [->, thick] (object.four west) to [out=180, in=180]
	(vnode.one west);
      \node [name=file, draw, rounded corners, fill=gray, shading=axis,
	     below=of vnode] {a file};
      \draw [->, thick] (vnode) -- (file);
      \node [name=page1, page, below right=of object] {vm\_page};
      \node [name=page2, page, node distance=0, below=of page1] {vm\_page};
      \node [name=page3, page, node distance=0, below=of page2] {vm\_page};
      \draw [pointer] (object.five east) -| (page1.north);
      \only<2> {
        \node [name=page4, page, color=red, node distance=0, below=of page3]
	   {vm\_page};
	\draw [->, thick, color=red] (file.east) to [out=0, in=270]
	  node [pos=.3, above, sloped] { page in } (page4.south);
      }
      \only<3> {
	\node [name=ell, draw, thick, color=red, ellipse,
	  minimum width=12ex, minimum height=2em] at (page2) {};
	\draw [->, thick, color=red] (ell.west) to [out=180, in=0]
	  node [below, sloped] { page out } (file.east);
      }
  \end{tikzpicture}
  \end{figure}
\end{column}
\begin{column}{.4\paperwidth}
  \begin{itemize}
    \item{Vnode object backs vm\_map\_entry made for program text, initialized
	  data, dynamic library text/data or result of mmap() on a file}
    \onslide<2-> {
      \item{Page fault brings a page from the file into memory}
    }
    \onslide<3-> {
      \item{If mapping is {\color{red}writable and shared},
	    dirty pages are written back to the file}
    }
  \end{itemize}
\end{column}
\end{columns}
\end{frame}


\FootReferences{}{sys/vm/vm\_object.h, sys/vm/vm\_object.c}
\begin{frame}
\frametitle<1>{Private mappings}
\frametitle<2>{Private mappings: a process fork(2)s}
\frametitle<3>{Private mappings: child modifies a page}
\frametitle<4-5>{Private mappings: shadow chains can grow hairy}
\frametitle<6-7>{Private mappings: shadow chain split}
\frametitle<8-9>{Private mappings: shadow chain collapse}
\begin{figure}
\begin{tikzpicture}[node distance=2mm]
  \node [name=procA, vmmap] { process A };
  \node [name=entA, vmentry, below right=of procA] { vm\_map\_entry };
  \draw [pointer] (procA.south) |- (entA.west);

  \node [name=obj, vmobject, below right=2mm and .4\paperwidth of entA]
	{ vnode object };
  \draw [pointer] (entA.east) -| (obj.north);
  \node [name=page1, page, below=of obj] {page 1};
  \draw [pointer] (obj) -- (page1);
  \node [name=page2, page, node distance=0, below=of page1] {page 2};
  \node [name=page3, page, node distance=0, below=of page2] {page 3};

  \onslide<2-7> {
    \node [name=procB, vmmap, below=.2\paperheight of procA] { process B };
    \node [name=entB, vmentry, below right=of procB] { vm\_map\_entry };
    \draw [pointer] (procB.south) |- (entB.west);
  }

  \only<2> {
    \draw [pointer] (entB.north) |- (node cs:name=obj, angle=200);
  }

  \onslide<3-> {
    \node [name=shadow, vmobject, above right=2mm and 10mm of entB]
	{ shadow object };
    \draw [pointer] (shadow.east) to [out=0, in=180] (obj.west);
    \node [name=pageB1, page, below=of shadow] {page 1};
    \draw [pointer] (shadow) -- (pageB1);
  }
  \onslide<3-7> {
    \draw [pointer] (entB.north) |- (shadow.west);
  }

  \onslide<4-> {
    \node [name=procC, vmmap, below=.2\paperheight of procB] { process C };
    \node [name=entC, vmentry, below right=of procC] { vm\_map\_entry };
    \draw [pointer] (procC.south) |- (entC.west);
  }

  \only<4> {
    \node [name=mark, above right=of entC] {} ;
    \draw [pointer] (entC.north)
	.. controls +(0,10mm) and (mark) .. (shadow.south west);
  }

  \onslide<5-8> {
    \node [name=shadowC, vmobject, below=.2\paperheight of shadow]
	{ shadow object };
    \draw [pointer] (entC.north) |- (shadowC.west);
  }
  \onslide<5,8> {
    \node [name=pageC2, page, below=of shadowC] {page 2};
    \draw [pointer] (shadowC) -- (pageC2);
  }
  \onslide<5,6,8> {
    \draw [pointer] (shadowC.north west)
	to [out=135, in=225] (shadow.south west);
  }

  \onslide<6-7> {
    \node [name=pageC1, page, below=of shadowC] {{\color{red}page 1}};
    \draw [pointer] (shadowC) -- (pageC1);
  }

  \onslide<7> {
    \draw [pointer, color=red] (shadowC.east)
	to [out=0, in=225] (obj.south west);
  }

  \onslide<8> {
    \node [name=procB, vmmap, below=.2\paperheight of procA, dotted]
	{ process B };
    \node [name=entB, vmentry, below right=of procB, dotted]
	{ vm\_map\_entry };
    \draw [pointer, dotted] (procB.south) |- (entB.west);
    \draw [pointer, dotted] (entB.north) |- (shadow.west);
  }

  \onslide<9> {
    \draw [pointer, color=red] (entC.north) to [out=90, in=180] (shadow.west);
    \node [name=pageB2, page, node distance=0, below=of pageB1]
	{{\color{red}page 2}};
  }
\end{tikzpicture}
\end{figure}
\end{frame}


\FootReferences{malloc(9),memguard(9),redzone(9)}{sys/kern/kern\_malloc.c}
\begin{frame}
\frametitle{kernel malloc(9)}
\begin{figure}
\begin{tikzpicture}[start chain=main going below,
                    node distance=3mm,
                    every join/.style={->, draw},
                    font=\scriptsize]
  \tikzset {
    base/.style={draw, thick, on chain, align=center, minimum height=4ex},
    proc/.style={base, rectangle},
    test/.style={base, diamond, aspect=3},
    term/.style={proc, rounded corners},
  }

  \node [name=entry, proc] {malloc(size)};
  \node [name=memguard, test, join] {memguard?};
  { [start branch=memguard going below]
    \node [name=guardalloc, proc, join] {v = memguard\_alloc()};
  }
  \node [name=pagesize, continue chain=going right, test, join]
	{size <= page size};
  { [start branch=large going right]
    \node [name=largealloc, proc, join] {v = uma\_large\_malloc()};
  }
  \node [name=zone, continue chain=going below, proc, join]
	{zone = min(power2(size))};
  \node [name=zalloc, proc, join] {v = uma\_zalloc(zone)};
  \node [name=mtp, proc, join] {malloc type accounting};
  \node [name=redzone, test, join] {redzone?};
  { [start branch=redzone going right]
    \node [name=redsetup, proc, join] {redzone\_setup(v)};
  }
  \node [name=return, term, join] {return v};

  \path (memguard.east) to node [yshift=1ex, red] {$no$} (pagesize);
  \path (memguard.south) to node [xshift=1em, green] {$yes$} (guardalloc);
  \draw [->,rounded corners] (guardalloc.south) |- (return.west);
  \path (pagesize.south) to node [xshift=1em, green] {$yes$} (zone);
  \path (pagesize.east) to node [yshift=1ex, red] {$no$} (largealloc);
  \draw [->,rounded corners] (largealloc.south) |- (mtp.east);
  \path (redzone.south) to node [xshift=1em, red] {$no$} (return);
  \path (redzone.east) to node [yshift=1ex, green] {$yes$} (redsetup);
  \draw [->,rounded corners] (redsetup.south) |- (return.east);
\end{tikzpicture}
\end{figure}
\end{frame}


\FootReferences{uma(9),vmstat(8)}{sys/vm/uma\_int.h,sys/vm/uma\_core.c}
\begin{frame}
\frametitle{uma(9): the zone allocator}
Let's look at malloc(9) zones in UMA:
\shellcmd{%
\# vmstat -z \textbar~head -1 \&\& vmstat -z \textbar~egrep '\^~\lbrack0-9\rbrack+:'\\
\begin{tabular}{rrrrrrrr}
ITEM  &   SIZE  &  LIMIT  &  USED  &  FREE  &    REQ &FAIL &SLEEP \\
16:   &     16, &      0, &  8055, &   479, &  30823,&   0,&   0 \\
32:   &     32, &      0, &   706, &  1044, & 197057,&   0,&   0 \\
64:   &     64, &      0, &  1905, &  8759, &3215548,&   0,&   0 \\
128:  &    128, &      0, &  1463, &  5202, & 387198,&   0,&   0 \\
256:  &    256, &      0, &   616, &  4304, & 757799,&   0,&   0 \\
512:  &    512, &      0, &   426, &  3846, &  50371,&   0,&   0 \\
1024: &   1024, &      0, &    80, &   116, &  70702,&   0,&   0 \\
2048: &   2048, &      0, &    79, &    87, &  71207,&   0,&   0 \\
4096: &   4096, &      0, &   193, &    49, &  78923,&   0,&   0 \\
\end{tabular}
}
\end{frame}

%
% This ugly TikZ code is taken from my master thesis and needs to
% be redesigned. Too much absolute node positioning here.
%
\newcommand{\structline}[2]{ #1 & #2 \\ }
\FootReferences{uma(9)}{sys/vm/uma\_int.h, sys/vm/uma\_core.c}
\begin{frame}
\frametitle{uma(9) is a slab allocator}
\begin{columns}
\begin{column}{.03\paperwidth}
\end{column}
\begin{column}{.97\paperwidth}
\begin{tikzpicture}[thick]
\draw [rounded corners] (1.0cm, 0cm) -- (0cm, 0cm) -- (0cm, 3cm)
        -- (1.0cm, 3cm);
\foreach \i in { 1, 2, 4, 5}
  \node (item\i) at (\i cm * 1.0, 0cm) [draw, anchor=south west,
   outer sep=0pt, minimum height=3cm, minimum width=1.0cm] {};
\node (item3) at (3.0cm, 0cm) [anchor=south west,
   outer sep=0pt, minimum height=3cm, minimum width=1.0cm] { \ldots };
% waste
\draw [pattern=checkerboard light gray]
        (node cs:name=item5, anchor=north east) --
        (7cm, 3cm) -- (7cm, 0cm) --
        (node cs:name=item5, anchor=south east) -- cycle;
% slab header
\shadedraw [shading=axis,shading angle=45] (7cm, 0cm) -- (7cm, 3cm)
        { [rounded corners] -- (12cm, 3cm) -- (12cm, 0cm) } -- cycle;
\draw (9.5cm, 1.5cm) node (slab) [rectangle split, rectangle split parts = 2,
				   rounded corners, draw, fill=white] {
	\textbf{struct uma\_slab}
	\nodepart{two}
	\begin{tabular}{ll}
	\structline{uint8\_t}   {*us\_data}
	\structline{slabbits}   {us\_free}
	\ldots
	\end{tabular}
};
% slab size
\draw (12cm / 2, 3.25cm) node (sizelabel) {slab size (PAGE\_SIZE)};
\draw [<-] (0cm, 3.25cm) -- (sizelabel.west);
\draw [->] (sizelabel.east) -- (12cm, 3.25cm);
% pointer
\draw [fill] (node cs:name=slab,angle=5) circle (2pt);
\draw [rounded corners, ->] (node cs:name=slab,angle=5) -| ++(0.25cm, -2cm) -|
        (0.55cm,0cm);
\end{tikzpicture}
\end{column}
\end{columns}
\end{frame}


\begin{frame}
\frametitle{uma(9) structures}
\begin{columns}
\begin{column}{.03\paperwidth}
\end{column}
\begin{column}{.97\paperwidth}
m = uma\_zalloc(zone, \ldots);
\begin{tikzpicture}[node distance=5mm]
  \node [name=zone, struct, rectangle split parts=4] {
    \textbf{struct uma\_zone}
    \nodepart{two} struct uma\_cache uz\_cpu\lbrack$ncpus$\rbrack
    \nodepart{three} LIST\_HEAD(,uma\_bucket)  uz\_buckets
    \nodepart{four} struct uma\_keg *uz\_keg
  };
  \node [name=cache, below left=5mm and 0mm of zone,
         struct, rectangle split parts=3] {
    \textbf{struct uma\_cache}
    \nodepart{two} uma\_bucket\_t    uc\_freebucket
    \nodepart{three} uma\_bucket\_t    uc\_allocbucket
  };
  \node [name=bucket, below=of zone, struct, rectangle split parts=4] {
    \textbf{struct uma\_bucket}
    \nodepart{two} int16\_t ub\_cnt
    \nodepart{three} int16\_t ub\_entries
    \nodepart{four} void *ub\_bucket\lbrack\rbrack
  };
  \draw [pointer] (zone.two west) -| (cache.north);
  \draw [pointer] (cache.two east) to [out=0, in=180] (bucket.one west);
\end{tikzpicture}
\end{column}
\end{columns}
\end{frame}

\end{document}
